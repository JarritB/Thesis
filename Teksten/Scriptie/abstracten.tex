\textbf{\underline{Nederlands}}
\par
Omwille van de complexe structuren die kristallen soms hebben, is in kristallografie de vaardigheid om kristallen te visualiseren belangrijk voor het delen van kennis. De software die op dit moment gebruikt worden om kristallen te visualiseren is geavanceerd. Deze hebben echter niet de capaciteiten die hedendaagse grafische software en game-engines voorzien. Eén programma dat flexibel en krachtig is op het vlak van grafisch ontwerp is Blender. Blender is een open source 3D computergrafisch softwarepakket dat gebruikt wordt voor 3D modellering, animaties en als game-engine. In deze thesis onderzoeken we de mogelijkheid tot het gebruiken van Blender als kristallografische visualisatietool door een interface te maken tussen het CIF-formaat en Blender, als basis voor toekomstige uitbreidingen. Door gebruik te maken van de Blender API en Python zijn we in staat een add-on te ontwerpen waarmee een gebruiker CIF-bestanden kan inlezen en eenheidscellen kan visualiseren. We merken echter dat het tekenen van kristallen in Blender lang duurt in vergelijking met andere kristalvisualisatiesoftware, zoals VESTA. We nemen waar dat het tekenen van eenheidscellen met een klein aantal atomen in Blender tot 14 maal langer duurt dan in VESTA, en voor eenheidscellen met een groot aantal atomen dit 400 maal zo lang duurt. We zien dat ongeveer 80\% van de loopduur van het programma besteed wordt aan het tekenen van objecten. Dit wil echter niet zeggen dat Blender ongeschikt is om kristallen te visualiseren. De flexibiliteit van de Blender API, in combinatie met het groot aantal mogelijkheden dat Python bied, maakt van Blender, hoewel een tragere, een krachtige kristallografische visualisatietool. 

\newpage
\textbf{\underline{English}}
\par
In crystallography, due to the complex structure crystals may have, the ability to visualise them is an important tool for sharing knowledge. The software that currently is used to visualise these crystals, already are quite advanced. However, none of these has the full range of 3D virtual reality options current graphics software and game engines provide. One program being extremely flexible and powerful in terms of graphical design is Blender. Blender is an open source 3D computer graphic software, and is mainly used for 3D modelling, animations and as a game engine. In this thesis we explore the possibility of using of Blender as a crystallographic visualisation tool by creating an interface between the CIF format and Blender as a base for further expansion. By using the Blender API and Python we are able to create an add-on with which a user can read in CIF files and draw unit cells in Blender. We find, however, that the drawing these crystals in Blender takes quite a while in comparison with other crystal visualisation tools, such as VESTA. We observe that for unit cells consisting of a rather small number of atoms, Blender takes about 14 times longer to draw this than VESTA, while for larger unit cells Blender can take up to 400 times longer to draw the unit cell. We find that around 80\% of the total runtime of the program is spent on drawing the objects. However, this does not necessarily mean that Blender is not suited for visualising crystals. The flexibility of the Blender API in combination with the range of possibilities Python gives, makes Blender, although slower, a very powerful crystallographic visualisation tool.                