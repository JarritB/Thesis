%%%%%%%%%%%%%%%%%%%%%%%%%%%%%%%%%%%%%%%%%%%%%%%%%%%%%%%%%%%%%%%%%%% 
%                                                                 %
%                           INLEIDING                             %
%                                                                 %
%%%%%%%%%%%%%%%%%%%%%%%%%%%%%%%%%%%%%%%%%%%%%%%%%%%%%%%%%%%%%%%%%%% 


\chapter{Inleiding}
Dit hoofdstuk omschrijft de context waarin deze thesis zich zal afspelen en vormt het fundament van deze scriptie. Om de rest van deze thesis te begrijpen is het van belang dit hoofdstuk grondig te lezen. 
\\
De eerste sectie van dit hoofdstuk schetst onder welke onderzoeksdomeinen deze thesis kan worden ondergebracht. De tweede sectie beschrijft het probleem waarop deze thesis een oplossing zal trachten te bieden. Het algemene doel van deze thesis is het vinden en uitwerken van deze oplossing, dit doel zal verder worden opgedeeld in verschillende, kleinere, doelstellingen. In sectie vier worden deze doelstellingen geformuleerd als onderzoeksvragen  bestaande uit een hoofdvraag en enkele deelvragen. Ten slotte is er nog een vijfde sectie waarin de hoofdstukken van deze scriptie worden beschreven. In een laatste sectie wordt dit hoofdstuk samengevat in een conclusie.

\section{Situering van het onderzoek}
Kristallografie is een tak van de wetenschap met als hoofdrol een eerder klein object, een kristal. Kristallen kunnen gezien worden als een puzzel van atomen. De stukjes van zo een puzzel kunnen verschillende chemische elementen zijn, maar evengoed allemaal dezelfde. Wat de kristallen zo uniek maakt, is hoe deze zijn opgebouwd. De opbouw bepaalt allerhande eigenschappen van dat bepaalde kristal. De complexiteit van kristallen kan oplopen tot op het punt waarop het bijna onmogelijk wordt deze in woorden te beschrijven en zelfs nog moeilijker deze te interpreteren. Dit is waar computertechnologie van pas komt. 
\\
Sinds enkele decennia zijn wetenschappers in staat kristalstructuren om te zetten in een digitaal formaat dat leesbaar is door zowel mensen als computers, het CIF-formaat (zie later). Dankzij dit standaardformaat is het mogelijk geworden de structuur van kristallen te lezen en te delen zonder kans op misinterpretatie, en biedt het computers de mogelijkheid zelfs de meest complexe kristalstructuren naar een driedimensionaal beeld om te zetten. Het 3D visualiseren van kristallen geeft wetenschappers meer inzicht en geeft meer mogelijkheden om hun kennis over te brengen. Dit alles heeft geleid tot een nauwe samenwerking tussen kristallografie en computerwetenschappen om de wereld van de kristallografie tot leven te brengen.


\section{Probleemstelling}
De programma’s die op dit moment door wetenschappers worden gebruikt bij het 3D visualiseren van kristallen bieden nog lang niet de vrijheid en aanbod aan features welke sommige hedendaagse 3D software en game-engines te bieden hebben. Het van de grond opbouwen van 3D visualisatiesoftware of zelfs reeds bestaande software aanpassen is een enorme taak en slechts weinig wetenschappers hebben hier de tijd noch de technische knowhow voor. Daarnaast is het moeilijk om programmeurs te vinden die de nodige kennis bezitten over kristallografie en kristalstructuren of zich hierin willen verdiepen.  
\\
Het gebruiken van het tekenprogramma Blender om kristallen te visualiseren is een stap in de goede richting. Deze open source software laat, mits enige kennis van het programma, toe driedimensionale figuren te creëren en te bekijken. Het groot aantal features in Blender biedt de gebruiker veel vrijheid aan, wat ontbreekt in hedendaagse kristalvisualisatiesoftware. 3D objecten aanmaken en bekijken doet men via de grafische interface van Blender of met behulp van een script. Het gebruiken van scripts is erg interessant voor de gebruiker omdat deze hiermee langdradige of repetitieve taken kan laten uitvoeren. 
\\
In de context van kristallografie laat dit toe dat een kristal, bestaande uit een groot aantal atomen, niet meer manueel moet getekend worden. Helaas biedt dit geen volledige oplossing voor het probleem, voor elk kristal moet er nog steeds een apart script worden geschreven waarin alle informatie van dat kristal staat. Het scripten in Blender wordt gedaan aan de hand van Python en de API van Blender (zie verder). Hierdoor is het schrijven van scripts niet vanzelfsprekend en zonder enige voorkennis onbegonnen werk.

\section{Doelstelling}
Het algemene doel van dit werk is het ontwerpen van een interface die kristalstructuren kan visualiseren in het voornoemde open source programma, Blender. Eens dit bereikt is, zijn de mogelijkheden virtueel eindeloos. 
\\
Een van de problemen bij het visualiseren van een kristal met scripten in Blender is dat voor elke kristalstructuur een nieuw script moet worden geschreven. De interface moet gezien worden als een black box met als input de beschrijving van een kristalstructuur en de driedimensionale voorstelling hiervan als output. Dit stelt de nood aan een inputformaat dat interpreteerbaar is door een computer. In eerste instantie zal er enkel worden gezocht naar het meest praktische formaat, om de omvang van het programma te beperken. Dit kan nadien nog uitbereid worden zodat verschillende formaten kunnen worden ingelezen. De eerste doelstelling is dus een keuze te maken van het formaat dat er kan ingelezen worden door de interface.  
\\
De volgende stap in het ontwerpproces van de interface is het schrijven van een functie die in staat is het eerder gekozen formaat in te lezen en om te zetten naar bruikbare data. Afhankelijk van de complexiteit en striktheid van het formaat kan het ontwerpen van dit soort routines erg tijdrovend zijn. Het is mogelijk dit te vermijden door op zoek te gaan naar reeds bestaande Python modules met een gelijkaardige werking en deze, indien mogelijk, te implementeren in de interface. Eens het formaat kan worden ingelezen, moet de nodige informatie worden opgeslagen. Python biedt de mogelijkheid om gebruik te maken van klassen, wat toelaat de kristaldata op te slaan in zelfgecreëerde datastructuren, wat de uiteindelijke dataverwerking zal vereenvoudigen. 
\\
De Blender API geeft de gebruiker een keuze uit een immens aantal functies. Dit zorgt er echter voor dat scripten in Blender zonder de nodige voorkennis erg complex kan worden. Een belangrijk onderdeel van deze scriptie is dan ook het zich verdiepen in de API van Blender en alle mogelijkheden die deze biedt. Met de nodige kennis van de functies wordt het mogelijk de gemaakte datastructuren driedimensionaal te visualiseren in Blender. Enkele basisfeatures die de interface moet hebben is een manier om elementen te onderscheiden aan de hand van hun kleur, automatisch bindingen maken tussen atomen op basis van hun onderlinge afstand en meerdere eenheidskristallen naast elkaar kunnen tekenen.
\\
Ten slotte zullen de limieten van Blender getest worden in de context van het wetenschappelijk onderzoek rond kristallen.  


\section{Onderzoeksvragen}
De doelstellingen uit vorige sectie kunnen worden geformuleerd als onderstaande onderzoeksvragen.
\\
Hoofdvraag:
\\
Is het mogelijk een interface te ontwerpen die in staat is kristalstructuren driedimensionaal te visualiseren in Blender?
\\
Deelvragen:
\\
Wat is de meest efficiënte methode om een kristalstructuur om te zetten in verwerkbare data?
\\
Wat is de beste manier om kristaldata te visualiseren in Blender?
\\
Is Blender een bruikbaar alternatief voor huidige kristalvisualisatiesoftware?
\\
Als uitbreiding kan volgende deelvraag nog worden onderzocht:
\\
Hoe kan het programma verder worden uitgebreid binnen de context van kristallografie? 


\section{Structuur van de tekst}
Het eerste hoofdstuk geeft een inleiding tot het algemene onderwerp van deze thesis. Hier zal onder andere de probleemstelling en het doel van dit onderzoek worden besproken. Dit hoofdstuk zal een kort overzicht te geven over de verdere inhoud van deze scriptie.  
\\
In het tweede hoofdstuk wordt de literatuurstudie besproken. Dit onderdeel van de tekst verdiept zich in kristallografie, het gebruikte formaat van kristalbeschrijving, reeds bestaande visualisatie software, de parser en tot slot Blender en de Blender API. Het doel van dit hoofdstuk is inzicht geven in het theoretische aspect van de thesis en de nodige kennis verschaffen over de gebruikte technologieën.
\\
In het derde hoofdstuk wordt het ontwerpproces beschreven. Hier kunnen de stappen worden gevolgd die zijn ondernomen in het opbouwen van de interface. Er wordt op een oppervlakkige manier gekeken naar de opbouw van het programma om het globale overzicht te behouden. Dit hoofdstuk schetst de algemene structuur van het programma en de gedachtengang tijdens het ontwerpen hiervan.
\\
De inhoud van het vierde hoofdstuk beschrijft de structuur van de ontworpen interface en hoe deze juist werkt. Deze tekst geeft een technische kijk op het programma en overloopt bepaalde onderdelen van de geschreven code in meer detail. Er wordt onder andere dieper ingegaan op de werking van de Blender API en de CIF-parser.
\\
Hoofdstuk vijf kijkt in detail naar de output van het programma. Er wordt dieper ingegaan op de resultaten, complicaties, mijlpalen en gemaakte fouten. 
De eindconclusie van de thesis wordt besproken in hoofdstuk zes. Alle relevante informatie uit voorgaande hoofdstukken wordt hier opgesomd om een duidelijk overzicht te geven over het volledige onderzoek. De resultaten worden in dit hoofdstuk nogmaals kort overlopen om zo tot een uiteindelijk besluit te komen betreffende dit eindwerk.
\\


\section{Conclusie}
Dit hoofdstuk gaf een overzicht over het onderwerp en de omvang van deze thesis, kaartte de probleemstelling aan en formuleerde enkele onderzoeksvragen die als rode draad doorheen deze tekst dienen. Met de kennis uit dit hoofdstuk zal de stap naar zowel de literatuurstudie als naar het meer praktische onderdeel van dit onderzoek minder groot zijn.
