In the world of crystallography, which encompasses the study of crystals, it can sometimes be nearly impossible to comprehensively describe the structure of a crystal on a paper, let alone in words, due to the sheer complexity of these structures. This is why researchers have been using software which is able to visualise the structure of crystals to share their knowledge. Despite the amount of data these programs can already display nowadays, none of them have the full range of 3D virtual reality options current graphic software packages and game engines provide. In this thesis we have focused on one of these in particular, Blender.
\par
Blender is an open source 3D computer graphic software tool which is mainly used for 3D modelling, creating animations and even as a game engine for 3D video games. Blenders makes use of it's own Blender Python API which is, like the software, free to use, and very well documented.
\par
The main goal of this thesis was to firstly see whether it is possible to develop a program that can visualise crystals in Blender. And how this compares to the crystallographic visualisation software which are currently used.
\par
We divided this goal into three general parts: researching the state of the art and programs that might be helpful, actually developing the program and lastly testing the program and analysing the results. 
\par
To get a grasp on what we want to visualise we explored the world of crystallography, and more specific, the structure of these crystals. We found that every crystal is build up out of unit cells. A unit cell is a 3D lattice in which the complete structure of a crystal resides. No matter how big a crystal is, it will always consist of a number of these identical unit cells. The shape of such a unit cell is always described by a number of parameters. The first six of these are called the lattice parameters. Three of which describe the length of the axis of the unit cell, while the others describe the angle between them.  