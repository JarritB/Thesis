%%%%%%%%%%%%%%%%%%%%%%%%%%%%%%%%%%%%%%%%%%%%%%%%%%%%%%%%%%%%%%%%%%% 
%                                                                 %
%                           INLEIDING                             %
%                                                                 %
%%%%%%%%%%%%%%%%%%%%%%%%%%%%%%%%%%%%%%%%%%%%%%%%%%%%%%%%%%%%%%%%%%% 


\chapter{Werking van het programma}

In hoofdstuk drie wordt een algemeen overzicht gegeven van de opbouw en werking van het programma. Er wordt echter niet dieper ingegaan op de onderliggende code van het programma. Dit hoofdstuk zal zich voornamelijk verdiepen in deze code, de structuur ervan en zullen de belangrijkste onderdelen ervan worden uitgelegd. Hiernaast zal er ook worden beschreven hoe de externe programma's moeten worden geïnstalleerd en hoe deze in het programma worden geïmplementeerd. 
\par
Om de gedachtegang achter het ontwerpproces makkelijker volgbaar te maken zal er in dit hoofdstuk vooral gebruikt gemaakt worden van een actieve schrijfwijze en de wetenschappelijke 'we'-vorm. De code zal worden uitgelegd aan de hand van listings die rechtstreeks uit de code van het programma komen. De volledige code kan worden teruggevonden in de bijlages van deze tekst.[Bijlage D]
\par


\section{Installatie van externe programma's}
Zoals in hoodstuk drie wordt beschreven gaan we gebruik maken van twee externe programma's die we zullen nodig zullen hebben om het programma uit te kunnen voeren, OpenBabel en PyCIFRW. Deze sectie legt uit hoe we deze kunnen installeren op een systeem. Hoewel deze in het geval van dit onderzoek met Windows 10 wordt gewerkt, werken deze software ook op oudere versies van Windows. Ons programma is gericht op besturingssysteemonafhankelijkheid, dit wil zeggen dat het ook op Linux en Apple moet werken. Het is echter mogelijk dat het installeren van deze externe programma's anders verloopt dan op Windows. De installatie van deze programma's zal in deze tekst echter enkel voor Windowssytemen worden uitgelegd.

\subsection{OpenBabel}  
De installatie van OpenBabel kan gevonden worden op hun webpagina: \url{http://openbabel.org/wiki/Category:Installation} en is gratis voor iedereen. Op deze pagina gaan we de OpenBabelGUI \textit{installer} downloaden, zie Figuur[4.1], let op dat de bit-versie met die van ons systeem overeenkomt. De download zou automatisch moeten starten. Ten slotte kunnen we de installatiewizard starten door het gedownloade bestand uit te voeren.
\par
Na het volgen van de installatiewizard zou de OpenBabel GUI moeten geïnstalleerd zijn op ons systeem, en kan het gebruikt worden in ons programma. Dit zal in het verdere verloop van dit hoofdstuk worden uitgelegd.

\subsection{PyCIFRW}
Voor de PyCIFRW module op ons systeem te installeren hebben we nood aan een prompt waarop Pyhton3.7 kan worden uitgevoerd. Zo'n prompt kan verkregen worden door anaconda te installeren en gebruik te maken van \textit{Anaconda prompt}. Vervolgens hebben we de pip installer nodig voor Python en is automatisch aanwezig op Python3.7
\par 
Pip laat ons toe allerlei Python modules te installeren, waaronder de PyCIFRW module. Dit doen we met door volgend commando in te geven in het prompt:
\begin{lstlisting}[caption=,numbers=none,language= bash]
  pip install PyCifRW
\end{lstlisting}
Dit zorgt ervoor dat we nu toegang hebben tot de PyCIFRW module wanneer we Python uitvoeren en kan eenvoudig getest worden door Python op te starten en het volgende lijn code uit te voeren:
\begin{lstlisting}[caption=,numbers=none]
  import CifFile
\end{lstlisting}

Als er geen foutmelding wordt gegeven wilt het zeggen dat de module correct is geïnstalleerd op voor onze Python installatie. 
\par
Doordat Blender een eigen Python installatie heeft zal deze nog geen toegang hebben tot onze geïnstalleerde modules. Dit kunnen we oplossen door de map die de PyCIFRW module bevat te kopiëren naar de Python installatie van Blender. Deze map kunnen we vinden op plaats waar we Python hebben geïnstalleerd op ons systeem, in het geval dat Anaconda wordt gebruikt vinden we deze in de installatiemap van Anaconda in de submap \textit{site-packages} (Anaconda3 \textgreater \textgreater{} Lib \textgreater \textgreater{} site-packages).
Vervolgens moeten we de map \textit{CifFile} kopiëren naar de Python installatie van Blender (blender-2.80.0-git.3c8c1841d72-windows64 \textgreater \textgreater{} 2.80 \textgreater \textgreater{} python \textgreater \textgreater{} lib \textgreater \textgreater{} site-packages). 
\par
Een alternatieve methode om de PyCIFRW module werkende te krijgen op Blender, zonder nood te hebben aan een Python3.7 installatie, is door de \textit{CifFile} map rechtstreeks in de Pythoninstallatiemap van Blender te plaatsen. De \textit{CifFile} kan gedownload worden vanaf de GitHub pagina van deze thesis: \url{https://github.com/JarritB/Thesis}
  
\section{Inlezen van het bestand}

\subsection{Controleren van het bestand}
Vooraleer we het bestand kunnen omzetten met OpenBabel moeten we controleren of het gekozen bestand wel dergelijk een CIF-bestand is, dit zou anders voor problemen kunnen zorgen in het verdere verloop van het programma.
\par
Om dit te controleren moeten we nazien of ons bestand de extensie \textit{.cif} heeft, in sommige gevallen wordt de extensie met hoofdletters geschreven, hiermee dienen we dus ook rekening te houden. Omdat de bestandsnaam het volledige pad naar het bestand inhoudt moeten we enkel de vier laatste tekens overhouden. Dit wordt gedaan op lijn 725 van Listing[4.1].
\par 
De variabele \textit{ext} zal nu enkel de laatste vier tekens van de filenaam bevatten. Bij een cif bestand zullen deze laatste vier altijd \textit{.cif} zijn, we moeten de variabele dus hiermee vergelijken. Door de \textit{.lower()} methode op te roepen op de extensie zal deze altijd naar kleine letters worden omgezet, en hoeven we dus geen rekening te houden met hoofdletters. In het geval dat de extensie niet gelijk is aan \textit{.cif} gaan we een foutboodschap weergeven in de terminal en gaan we in de \textit{user\_feedback} variabele ook een boodschap zetten die zal verschijnen op onze add-on, deze variabele wordt later verder uitgelegd. Omdat er een fout bestand is ingegeven zal het programma beëindigd worden met een \textit{return}, het heeft geen zin dit bestand proberen te tekenen. In listing[4.1] wordt de controle van het bestand gedaan.
\lstinputlisting[linerange={726-730},firstnumber=726,caption=Controle van de extensie]{listings/__init__.py}

\subsection{Uitvoeren van OpenBabel}
Nu we zeker weten dat we met een CIF-bestand te maken hebben kunnen we de inwendige symmetrie van het kristal wegwerken met OpenBabel. 
\par
Het oproepen van OpenBabel doen we met behulp van de \textit{subprocess} module van python. Wanneer we deze module oproepen zal er een nieuw process worden gestart, die in ons geval OpenBabel zal uitvoeren.
\par
Voor we de \textit{subprocess} module kunnen oproepen in ons script moeten we  deze importeren met de lijn code in Listing[4.2]

\lstinputlisting[linerange={13-13},firstnumber=13,caption=Importeren van de subprocess module]{listings/__init__.py}

Vooraleer we OpenBabel gaan oproepen gaan we controleren of OpenBabel wel geïnstalleerd is op ons systeem. Dit wordt gedaan met een \textit{try-except} blok, zie Listing[4.3], welke gaat proberen de code in het \textit{try} gedeelte uit te voeren. Als dit niet lukt zal het programma in plaats daarvan het \textit{except} gedeelte uitvoeren. Bij een correcte installatie van OpenBabel zal de \textit{obabel\_fill\_unit\_cell} functie worden uitgevoerd, zie lijn 735 van Listing[4.3].
\par
In het geval dat OpenBabel niet kan worden opgeroepen gaan we een foutboodschap weergeven in de terminal en in de \textit{user\_feedback} variabele en zal ons programma beëindigd worden met een \textit{return}. Het kristal zou deels kunnen getekend worden zonder OpenBabel maar de conversie die OpenBabel doet lost nog een ander probleem op. Omdat dit probleem grotendeels te maken heeft met het parsen van het bestand zal het besproken worden in de volgende sectie.

\lstinputlisting[linerange={732-739},firstnumber=732,caption=Controle van de OpenBabel installatie]{listings/__init__.py}
\par	
De functie \textit{obabel\_fill\_unit\_cell} heeft het pad naar het ingevoerde CIF-bestan en de naam van het geconverteerde CIF-bestand als argumenten en bestaat uit slechts één lijn code. Op deze lijn roepen we de \textit{run} methode op de \textit{subprocess} module op, deze heeft als argument een \textit{string} waarin het uit te voeren commando staat zoals het in een prompt zou worden uitgevoerd. Het commando waarmee we OpenBabel uitvoeren bestaat uit volgende parameters:
\begin{itemize}
\item obabel: roept OpenBabel op
\item -icif: type van het invoerbestand, wordt gevolgd door bestandsnaam
\item -ocif: type van het uitvoerbestand
\item -O: gevolgd door naam van het uitvoerbestand
\item --fillUC: selecteert de mode waarin de symmetrie zal worden omgezet
\item keepconnect: parameter van de fillUC mode, tekent ook atomen buiten de eenheidscel
\end{itemize}
In onze code ziet het dan als volgt uit:
\lstinputlisting[linerange={671-671},firstnumber=671,caption=Uitvoeren van OpenBabel]{listings/__init__.py}

\subsection{Parsen met PyCifRW}



 





